\documentclass[a4paper,11pt,leqno,fleqn]{article}

% commands =====================================================================
\renewcommand{\baselinestretch}{1.25}\normalsize
\renewcommand{\figurename}{Fig.}
\newcommand{\vect}[1]{\mathbf{#1}}
\newcommand{\thin}{\thinspace}
\newcommand{\thick}{\thickspace}
\newcommand{\N}{\mathrm{N}} %Normal Distribution
\newcommand{\U}{\mathrm{U}} %Uniform Distribution
\newcommand{\D}{\mathrm{D}} %Dirichlet Distribution
\newcommand{\W}{\mathrm{W}} %Wishart Distribution
\newcommand{\E}{\mbox{E}}   %Expectation
\newcommand{\Iden}{\mathbb{I}}  %Identity Matrix
\newcommand{\Ind}{\mathrm{I}}   %Indicator Function
\newcommand{\var}{\mathrm{var}\thin}
\newcommand{\plim}{\mathrm{plim}\thin}
\newcommand{\cov}{\mathrm{cov}\thin}
\newcommand\indep{\protect\mathpalette{\protect\independenT}{\perp}}
\def\independenT#1#2{\mathrel{\rlap{$#1#2$}\mkern5mu{#1#2}}}
\renewcommand{\arraystretch}{1.05}
\newenvironment{boenumerate}
{\begin{enumerate}\renewcommand\labelenumi{\textbf{(\theenumi)}}}
{\end{enumerate}}

% packages ======================================================================

\usepackage{subcaption}
\usepackage{graphicx}

\usepackage{times}
\usepackage[T1]{fontenc}
\usepackage[utf8]{inputenc}
\usepackage[natbib=true,giveninits=false,uniquename=false,uniquelist=false,bibencoding=inputenc,bibstyle=authoryear-ibid,citestyle=authoryear-comp,maxcitenames=2,maxbibnames=10,useprefix=false,sortcites=true,backend=biber]{biblatex}
\AtBeginDocument{\toggletrue{blx@useprefix}}
\AtBeginBibliography{\togglefalse{blx@useprefix}}
\usepackage{setspace}
\usepackage[unicode=true]{hyperref}
\hypersetup{
    colorlinks=true,
    linkcolor=black,
    anchorcolor=black,
    citecolor=black,
    filecolor=black,
    menucolor=black,
    runcolor=black,
    urlcolor=blue}
\usepackage{threeparttable}
\usepackage{lscape}
\usepackage{float, afterpage, rotating, graphicx}
\usepackage{epstopdf}
\usepackage{longtable, booktabs, tabularx}
\usepackage{amsmath, amsfonts, amsthm, bm, amssymb}
\usepackage{fancyvrb, moreverb, relsize}
\usepackage{eurosym, calc, chngcntr}
\usepackage{caption}
\usepackage{mdwlist}
\usepackage{xfrac}
\usepackage{setspace}
\usepackage{adjustbox}
% \usepackage[position=bottom]{subfig}
\usepackage[usenames,dvipsnames]{xcolor}
\usepackage{fancybox}
\usepackage{appendix}
\usepackage{enumerate}
\usepackage{color,colortbl}
\usepackage[left=2.5cm, right=3.25cm, top=2cm, bottom=2cm]{geometry}
\usepackage{placeins}

% colors ======================================================================
% blues
\definecolor{lb}{HTML}{3498DB}
\definecolor{b}{HTML}{1565C0}
\definecolor{db}{HTML}{002080}
% greens
\definecolor{lg}{HTML}{B2EC5D}
\definecolor{g}{HTML}{55A868}
\definecolor{dg}{HTML}{2E7D32}
% yellows
\definecolor{ly}{HTML}{F9A825}
\definecolor{y}{HTML}{FF8F00}
\definecolor{dy}{HTML}{EF6C00}
% reds
\definecolor{lr}{HTML}{D84315}
\definecolor{r}{HTML}{C44E52}
\definecolor{dr}{HTML}{C62828}
% violet
\definecolor{lv}{HTML}{C71585}
\definecolor{v}{HTML}{9B59B6}
\definecolor{dv}{HTML}{6A1B9A}
% grey
\definecolor{ln}{HTML}{BDBDBD}
\definecolor{n}{HTML}{757575}
\definecolor{dn}{HTML}{616161}
% palettes
\definecolor{p1}{HTML}{4A6FAC}
\definecolor{p2}{HTML}{53A465}
\definecolor{p3}{HTML}{C04C50}
\definecolor{p4}{HTML}{7E6FAE}
\definecolor{p5}{HTML}{C8B571}
\definecolor{p6}{HTML}{62B1C8}
% brown
\definecolor{brown}{HTML}{4E342E}

% settings ====================================================================
\setlength{\skip\footins}{1.0cm}

\setlength{\parindent}{0mm}
\setlength{\parskip}{2mm}


% title =======================================================================
\title{Die Bedeutung individuellen Verhaltens über den Jahreswechsel für die Weiterentwicklung der Covid-19 Pandemie in Deutschland}
\date{18. Dezember 2020}
\author{Janoś Gabler, Tobias Raabe, Klara Röhrl, Hans-Martin von Gaudecker}

\addbibresource{refs.bib}

\begin{document}

% titlepage
\maketitle
\pagenumbering{arabic}
\setcounter{page}{1}


\section{Einleitung}

Um den steigenden Coronafallzahlen zu begegnen, wurde in Deutschland von November bis Mitte Dezember ein leichter Lockdown verhängt.
Dieser hat jedoch lediglich zu einem Stagnieren der Fallzahlen geführt.
Deshalb befindet sich Deutschland seit dem 16. Dezember erneut in einem harten Lockdown, um das Infektionsgeschehen wieder unter Kontrolle zu bringen.

Ein sehr wichtiger Faktor für den Erfolg der Maßnahmen ist, wie sich die Menschen in Deutschland über die Weihnachtsfeiertage verhalten.
Familienfeiern, wo Menschen intensiv Zeit miteinander verbringen, die sich sonst sehr selten treffen, bergen ein großes Ansteckungsrisiko.
Je häufiger und je mehr unterschiedliche Haushalte zusammen kommen, umso größer ist das Potential, dass neue Infektionsketten in Gang gesetzt werden.
Gleichzeitig ist damit zu rechnen, dass über die Feiertage geringere Testkapazitäten zur Verfügung stehen und die Gesundheitsämter noch weniger in der Lage sein werden, Kontaktnachverfolgung in der nötigen Geschwindigkeit zu betreiben.
Allerdings ist eine private Kontaktnachverfolgung von Familientreffen aufgrund der persönlichen Bekanntschaft viel leichter möglich, was bei konsequenter Kommunikation und Selbstquarantäne einen großen Teil in Gang gesetzte Infektionsketten frühzeitig ersticken könnte.

Mit Hilfe eines neu entwickelten mikrobasierten Modells simulieren wir die Entwicklung des Infektionsgeschehens in Deutschland für verschieden Szenarien, wie Menschen in Deutschland Weihnachten feiern und in welchem Maße sie in den darauffolgenden Tagen selbstständig Kontaktnachverfolgung und Selbstquarantäne betreiben.

\textcolor{red}{Kurze Zusammenfassung der Ergebnisse}

\section{Das Modell}

Das Modell von Gabler, Raabe und Röhrl\footnote{Eine detaillierte Beschreibung des Modells findet sich unter \url{https://covid-19.iza.org/publications/dp13899/}.} basiert auf agentenbasierten Simulationsmodellen, ersetzt aber die Kontakte zwischen sich bewegenden Teilchen durch Kontakte zwischen Personen, die arbeiten, zur Schule gehen, in einem Haushalt leben und Freizeitaktivitäten ausüben.

Es hat die folgenden Eigenschaften, die es sehr geeignet machen, um die Effekte von Politikmaßnahmen und individuellem Verhalten vorherzusagen:

\begin{enumerate}
    \item Im  Mittelpunkt des Modells steht die Begegnung von Menschen mit Menschen auf der Grundlage eines Matching-Algorithmus. Das Modell unterscheidet verschiedene Arten von Kontakten, wie beispielsweise Haushalte, Freizeitaktivitäten, Schulen, Kindergärten und verschiedene Arten von Kontakten am Arbeitsplatz. Die Kontaktarten können zufällig oder wiederkehrend sein und variieren in ihrer Häufigkeit and Ansteckungswahrscheinlichkeit.
    \item Das Modell kann entdeckte und unentdeckte Infektionen unterscheiden.\footnote{Hierfür nutzen wir Schätzungen des Dunkelziffer Radar Projekts: \url{https://covid19.dunkelzifferradar.de/}}
    \item Das Modell erlaubt individuelle Verhaltensanpassungen. Beispielsweise können Individuen, die Symptome entwickeln oder einen positiven Test erhalten, ihre Kontakte reduzieren.
    \item Politikmaßnahmen können im Modell so umgesetzt werden, dass Kontaktarten ganz oder teilweise abgeschaltet werden. Die Reduzierung von Kontakten kann zufällig oder systematisch erfolgen, z.B. können Arbeitskontakte so reduziert werden, dass nur noch systemrelevante Berufsgruppen Arbeitskontakte haben.
    \item Das mit der medizinischen Literatur und Kontaktzahlen aus früheren Erhebungen kalibrierte Modell erzielt mit wenigen frei geschätzten Ansteckungswahrscheinlichkeiten eine gute Übereinstimmung mit den deutschen Infektions- und Sterberaten.
\end{enumerate}

Zu den Hintergrundmerkmalen jedes simulierten Individuums im Modell gehören Alter, Landkreis und Beruf.
Kontaktmodelle sind Funktionen, die individuelle Merkmale auf eine vorhergesagte Anzahl von Kontakten abbilden.
Diese Anzahl an Kontakten wird durch einen Matching-Algorithmus in Infektionen übersetzt.
Es gibt verschiedene Matching-Algorithmen für wiederkehrende Kontakte (z.B. Klassenkameraden, Familienmitglieder) und nicht wiederkehrende Kontakte (z.B. Kunden, Kontakte in Supermärkten).
Die Infektionswahrscheinlichkeit kann für jeden Kontakttyp unterschiedlich sein.
Alle Arten von Kontakten können hinsichtlich geografischer und demografischer Merkmale assortativ sein, z.B. sind Arbeitskontakte häufiger mit Menschen aus dem gleichen Landkreis.
Das Krankheitsmodell berücksichtigt asymptomatische Fälle, leichte Symptome und Symptome, die Behandlung auf einer Intensivstation erfordern und Altersgradienten, bspw. in  den  Wahrscheinlichkeiten für schwere Krankheitsverläufe.

\section{Szenarien für den Jahreswechsel}

Wir variieren das Verhalten der Deutschen über zwei Dimensionen:

\begin{enumerate}
    \item \textbf{Die Anzahl und Intensität der Weihnachtstreffen.}
    Ein volles Ausreizen der aktuellen Regulierungen würde bedeuten, dass sich an jedem der drei Feiertage (24., 25. und 26.12.) jeder Haushalt mit je zwei weiteren, unterschiedlichen Haushalten trifft. Eine vorsichtigere Verhaltensweise, die wir simulieren, ist, dass die Menschen die Feiertage in festen Gruppen von drei Haushalten feiern. Um den Effekt der Häufigkeit der Treffen zeigen zu können, schauen wir uns zusätzlich den Fall an, wo sich feste Haushaltsgruppen nur am 24. und 25.12. treffen.
    \item \textbf{Der Grad der privaten Kontaktnachverfolgung.} Hier unterscheiden wir ein Szenario ohne private Kontaktnachverfolgung, wo Verwandte nicht über Symptome oder Testergebnisse informiert werden. In einem weiteren Szenario nehmen wir an, dass die private Kontaktnachverfolgung die Kontakte von Menschen mit Risikokontakten um 50\% reduziert und ein Szenario, wo sie die Kontakte von Menschen mit Risikokontakten um 90\% reduziert.
\end{enumerate}

Zusätzlich nehmen wir an, dass drei Viertel der Bevölkerung am 23.12. zusätzliche Kontakte durch Vorbereitungen haben. Dies simuliert den Effekt durch beispielsweise Zugfahrten und Weihnachtseinkäufe.

\FloatBarrier
\section{Ergebnisse}


\subsection{Der Effekt verschiedener Arten, Weihnachten zu feiern}

\begin{figure}
\label{fig:effect_of_christmas_without_contact_tracing}
\includegraphics[width=\textwidth]{../../bld/simulation/effect_of_christmas_mode_with_None_contact_tracing}
\caption{
    Das linke Panel zeigt die Entwicklung der beobachteten Fallzahlen, das rechte die Entwicklung der tatsächlichen Fallzahlen zwischen 2. Dezember 2020 und 10. Januar 2021. Ausgehend von keiner privaten Kontaktnachverfolgung wird das Infektionsgeschehen bei Weihnachtsfeiern mit wechselnden Haushalten (blau), im festen Kreis (gelb) und im festen Kreis im reduzierten Rahmen (rot) gezeigt.
}
\end{figure}





\FloatBarrier


\begin{figure}
\label{fig:effect_of_christmas_with_mediocre_contact_tracing}
\includegraphics[width=\textwidth]{../../bld/simulation/effect_of_christmas_mode_with_0.5_contact_tracing}
\caption{
    Das linke Panel zeigt die Entwicklung der beobachteten Fallzahlen, das rechte die Entwicklung der tatsächlichen Fallzahlen zwischen 2. Dezember 2020 und 10. Januar 2021. Ausgehend von privater Kontaktnachverfolgung, die die Kontakte von Menschen mit Risikokontakten um 50\% reduziert, wird das Infektionsgeschehen bei Weihnachtsfeiern mit wechselnden Haushalten (blau), im festen Kreis (gelb) und im festen Kreis im reduzierten Rahmen (rot) gezeigt.
}
\end{figure}

\FloatBarrier


\begin{figure}
\label{fig:effect_of_christmas_with_good_contact_tracing}
\includegraphics[width=\textwidth]{../../bld/simulation/effect_of_christmas_mode_with_0.1_contact_tracing}
\caption{
    Das linke Panel zeigt die Entwicklung der beobachteten Fallzahlen, das rechte die Entwicklung der tatsächlichen Fallzahlen zwischen 2. Dezember 2020 und 10. Januar 2021. Ausgehend von privater Kontaktnachverfolgung, die die Kontakte von Menschen mit Risikokontakten um 90\% reduziert, wird das Infektionsgeschehen bei Weihnachtsfeiern mit wechselnden Haushalten (blau), im festen Kreis (gelb) und im festen Kreis im reduzierten Rahmen (rot) gezeigt.
}
\end{figure}

\FloatBarrier
\subsection{Der Effekt von privater Kontaktverfolgung und Selbstquarantäne}


\begin{figure}
\label{fig:effect_of_contact_tracing_with_full_christmas}
\includegraphics[width=\textwidth]{../../bld/simulation/effect_of_private_contact_tracing_full}
\caption{
    Das linke Panel zeigt die Entwicklung der beobachteten Fallzahlen, das rechte die Entwicklung der tatsächlichen Fallzahlen zwischen 2. Dezember 2020 und 10. Januar 2021. Ausgehend von Weihnachtsfeiern mit wechselnden Haushaltsgruppen wird das Infektionsgeschehen bei anschließend nicht statt-findender (blau), 50 prozentiger (gelb) und 90 prozentiger (rot)
    privater Kontaktnachverfolgung und Kontaktreduktion durch Menschen als Reaktion auf Risikokontakte während der Feiertage gezeigt.
}
\end{figure}


\FloatBarrier



\begin{figure}
\label{fig:effect_of_contact_tracing_with_same_group_christmas}
\includegraphics[width=\textwidth]{../../bld/simulation/effect_of_private_contact_tracing_same_group}
\caption{
    Das linke Panel zeigt die Entwicklung der beobachteten Fallzahlen, das rechte die Entwicklung der tatsächlichen Fallzahlen zwischen 2. Dezember 2020 und 10. Januar 2021. Ausgehend von drei Weihnachtsfeiern mit gleichbleibenden Haushaltsgruppen wird das Infektionsgeschehen bei anschließend nicht statt-findender (blau), 50 prozentiger (gelb) und 90 prozentiger (rot)    privater Kontaktnachverfolgung und Kontaktreduktion durch Menschen als Reaktion auf  Risikokontakte während der Feiertage gezeigt.
}
\end{figure}


\FloatBarrier


\begin{figure}
\label{fig:effect_of_contact_tracing_with_meet_twice_christmas}
\includegraphics[width=\textwidth]{../../bld/simulation/effect_of_private_contact_tracing_meet_twice}
\caption{
   Das linke Panel zeigt die Entwicklung der beobachteten Fallzahlen, das rechte die Entwicklung der tatsächlichen Fallzahlen zwischen 2. Dezember 2020 und 10. Januar 2021. Ausgehend von zwei Weihnachtsfeiern mit gleichbleibenden Haushaltsgruppen wird das Infektionsgeschehen bei anschließend nicht statt-findender (blau), 50 prozentiger (gelb) und 90 prozentiger (rot)
   privater Kontaktnachverfolgung und Kontaktreduktion durch Menschen mit Risikokontakten
   während der Feiertage gezeigt.
}
\end{figure}


\FloatBarrier


\end{document}
